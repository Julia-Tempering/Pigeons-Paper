\section{Towards tempered Particle Gibbs for Turing-complete PPLs}

Particle Gibbs (PG, \citealp{andrieu2010particle}) is an MCMC
algorithm that has been shown \citep{wood2014new} to be a natural sampler for
arbitrary programs written in Turing-complete probabilistic programming 
languages (PPLs). For example, the Julia package Turing.jl \citep{ge2018turing}
provides a PG sampler that can handle any program written in DynamicPPL
\citep{tarek2020dynamicppl}, its underlying PPL.
However, this generality comes at the expense of poor performance in moderately
complex models. This is due to the fact that PG is, essentially, a smart 
algorithm for selecting samples from the prior that are close to the posterior.
When the prior is far from the posterior, this process can take a long time to
produce a sample of acceptable quality.

But this is where Pigeons can help: within a PT run, PG can be
used to explore each distribution in the annealing path. We expect PG to work 
more effectively on tempered 
distributions closer to the prior. Such samples can then be transported towards 
the posterior chain via the PT process.
In order to construct this sequence of distributions for an arbitrary program in
a Turing-complete PPL, it suffices to inject an inverse temperature parameter 
$\beta\in[0,1]$ in the call to compute the log density of every \texttt{observe} 
statement in the program. Leveraging multiple dispatch in Julia, we have
successfully applied this technique to DynamicPPL programs in our package
TPGExplorers.jl\footnote{\url{https://github.com/Julia-Tempering/TPGExplorers.jl}}.

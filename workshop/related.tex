\section{Related work}
Automated software packages for Bayesian inference have revolutionized Bayesian data 
analysis in the past two decades, and are now a core part of a typical applied statistics 
workflow. For example, packages such as BUGS 
\cite{lunn2013bugs, lunn2009bugs, lunn2000winbugs}, 
JAGS \cite{hornik2003jags}, Stan \cite{carpenter2017stan}, PyMC3 
\cite{salvatier2016probabilistic}, and Turing.jl \cite{ge2018turing} have been widely used in 
many scientific applications.

 
These software packages often provide two key user-facing components: (1) a 
probabilistic programming language (PPL), which allows users to specify Bayesian 
statistical models in code with a familiar, mathematics-like syntax; and (2) an 
inference engine, which is responsible for performing computational Bayesian
inference once the model and data have been specified. 
Pigeons.jl focuses on the development 
of a new inference engine that employs distributed, high-performance computing.


Inference engines available in existing software packages are varied in their 
capabilities and limitations. For instance, the widely-used Stan inference engine 
is only capable of handling real-valued (i.e., continuous) parameters. 
Other tools, such as JAGS, are capable of handling discrete-valued parameters but 
are limited in their capability to handle custom data-types 
(e.g. phylogenetic trees). 
Of those that have the capability to model arbitrary data types, 
none have an automatically distributed implementation to our knowledge.
Turing.jl \cite{ge2018turing} offers another popular inference engine and PPL, 
however Pigeons.jl allows one to interface with several different PPLs as well as 
to easily perform distributed computation.


There is also a vast literature on distributed and parallel 
Bayesian inference algorithms 
\cite{bardenet2017markov, brockwell2006parallel, calderhead2014general,
jacob2020unbiased, jacob2011using, lee2010utility, scott2016bayes, 
wang2015parallelizing, wu2017average, zhu2017big}.
These methods unfortunately do not lead to widely 
usable software packages because they either introduce unknown amounts of 
approximation error, involve significant communication cost, or reduce the 
generality of Bayesian inference. 

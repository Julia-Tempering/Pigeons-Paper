\section{Overview of Pigeons.jl}\label{sec:overview}

Pigeons.jl exploits the fact that the exploration phase in PT is embarrassingly parallel
to provide a multithreaded and distributed Julia implementation of the algorithm.

In many problems, e.g. in Bayesian 
statistics, the target $\pi$ is typically known only up to a normalizing constant, 
\[
\label{eq:normalizing_constant}
  \pi(x) = \frac{\gamma(x)}{Z}, \qquad Z = \int \gamma(x) \, \dee x,
\]  
where $\gamma$ can be evaluated pointwise but $Z$ is typically unknown.
Pigeons.jl takes as input the function $\gamma$.


The output of Pigeons.jl can be used for two main tasks:
\begin{enumerate}
    \item Approximating integrals of the form $\int f(x) \pi(x) \, \dee x$.  
    For example, the choice $f(x) = x$ computes the mean and 
    $f(x) = I[x \in A]$ computes the probability of $A$ under $\pi$,
    where $I[\cdot]$ denotes the indicator function.

    \item Approximating the value of the normalization constant $Z$. For 
    example, in Bayesian statistics, this corresponds to the 
    marginal likelihood, which can be used for model selection. 
\end{enumerate}
The Pigeons package particularly shines compared to traditional sampling approaches in the 
following scenarios:
\begin{itemize}
    \item When the target density $\pi$ is challenging due to a complex structure 
    (e.g., high-dimensional, multi-modal, etc.).
    
    \item When the user needs not only $\int f(x) \pi(x) \, \dee x$ but also
    the normalizing constant $Z$. 
    Many existing tools focus on the former and struggle or fail to do the latter. 
    
    \item When the target distribution $\pi$ is defined over a non-standard space, 
    e.g. a combinatorial object such as a phylogenetic tree.  
\end{itemize}

There are two fundamental concepts in the implementation that we will use
in the following sections. For a distribution $\pi_i$ in the path, we call its
index the $i$-th \emph{chain}. During execution, we assign \emph{replicas} to 
work on the distributions associated to the chains. The map between chains and
replicas is one-to-one, and thus can be seen as a permutation of the 
set $\{1,\dots,N\}$. Replicas are the fundamental unit of parallelization in PT:
they may be distributed across threads, processes, and machines. 

In \cref{app:examples} we provide examples of how to use Pigeons.jl. For more details
on the implementation, see \citet{surjanovic2025pigeons}.


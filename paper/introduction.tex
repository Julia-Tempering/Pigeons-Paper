\section{Introduction}
In many scientific application domains, the ability to obtain samples from a 
distribution is crucial. 
For instance, sampling methods have been used to discover magnetic polarization 
in the black hole of galaxy M87 \cite{akiyama2021seven}
and to image the Sagittarius A* black hole \cite{akiyama2022first}.
They have also been used to 
model the evolution of single-cell cancer genomes \cite{salehi2021clonal}, 
infer plasma dynamics inside nuclear fusion reactors \cite{gota2021overview}, 
and to identify gerrymandering in Georgia's 2021 congressional districting plan 
\cite{zhao2022mathematically}.
Similarly, evaluating high-dimensional integrals or sums over complicated 
combinatorial spaces are related tasks that can also be solved with sampling 
via Markov chain Monte Carlo (MCMC) methods. 
However, such calculations can often be bottlenecks in the scientific process, with 
simulations that can last days or even weeks to finish. 

\medskip
Pigeons.jl\footnote{The source code for Pigeons.jl can be found at 
\url{https://github.com/Julia-Tempering/Pigeons.jl}. 
Pigeons.jl v0.2.0, used in the example scripts below, is currently compatible with Julia 1.8+.}
enables users to sample efficiently from high-dimensional and complex distributions 
and solve integration problems by 
implementing state-of-the-art sampling algorithms \cite{syed2021nrpt,surjanovic2022vpt} 
that leverage distributed computation. Its simple API allows users to perform such computations 
single-threaded, multi-threaded, and/or distributed over MPI-communicating machines. 
We have stress-tested Pigeons successfully with up to 1000 machines running 
concurrently on a compute cluster (Sockeye) at the University of British Columbia.
Further, Pigeons comes with guarantees on strong parallelism invariance wherein 
the output for a given seed is \emph{identical} irrespective 
of the number of threads or processes. 
Such a level of reproducibility is rare in distributed software but of 
great use for the purposes of debugging in the context of sampling algorithms, 
which produce stochastic output.
Specifically, Pigeons.jl is designed to be suitable and yield reproducible output for:
\begin{enumerate}
    \item one machine running on one thread;
    \item one machine running on several threads;
    \item several machines running, each using one thread, and
    \item several machines running, each using several threads.
\end{enumerate}


\subsection{Problem formulation}
We describe the class of problems that can be approached using Pigeons.jl.
Let $\pi(x)$ denote a probability density
\footnote{This density may also be a probability mass function 
(discrete variables). We also allow for a combination of discrete/continuous variables.} 
called the \emph{target}. 
In many problems, e.g. in Bayesian statistics, the density $\pi$ is typically 
known only up to a normalizing constant, 
\[
\label{eq:normalizing_constant}
  \pi(x) = \frac{\gamma(x)}{Z}, \qquad Z = \int \gamma(x) \, \dee x,
\]  
where $\gamma$ can be evaluated pointwise but $Z$ is typically unknown.
Pigeons.jl takes as input the function $\gamma$.

\medskip 
The output of Pigeons.jl can be used for two main tasks:
\begin{enumerate}
    \item Approximating integrals of the form $\int f(x) \pi(x) \, \dee x$.  
    For example, the choice $f(x) = x$ computes the mean and 
    $f(x) = I[x \in A]$ computes the probability of $A$ under $\pi$,
    where $I[\cdot]$ denotes the indicator function.

    \item Approximating the value of the normalization constant $Z$. For 
    example, in Bayesian statistics, this corresponds to the 
    marginal likelihood, which can be used for model selection. 
\end{enumerate}
Its implementation shines compared to traditional sampling packages in the 
following scenarios:
\begin{itemize}
    \item When the target density $\pi$ is challenging due to a complex structure 
    (e.g., high-dimensional, multi-modal, etc.).
    
    \item When the user needs not only $\int f(x) \pi(x) \, \dee x$ but also
    the normalization constant $Z$. 
    Many existing tools focus on the former and struggle or fail to do the latter. 
    
    \item When the target distribution $\pi$ is defined over a non-standard space, 
    e.g. a combinatorial object such as a phylogenetic tree.  
\end{itemize}


\subsection{What is of interest to the general Julia developer?}
Ensuring code correctness at the intersection of randomized, parallel, and 
distributed algorithms is a challenge. To address this challenge, we designed Pigeons.jl 
based on a principle that we refer to as \textit{strong parallelism invariance}
(SPI).
Namely, the output of Pigeons.jl is completely invariant to the number of machines 
and/or threads.
Without explicitly keeping SPI in mind during software construction, 
the (random) output of the algorithm is only guaranteed to have the same distribution.
This is a much weaker guarantee that, in particular, makes debugging difficult.
However, with our notion of SPI we make debugging and 
software validation considerably easier. This is because the developer can 
first focus on the fully serial randomized algorithm, and then use it as an 
easy-to-compare gold-standard reference for parallel/distributed implementations. 
This strategy is used extensively in Pigeons.jl to ensure correctness. In contrast, 
testing equality in distribution, while possible (e.g., see \cite{geweke2004getting}), 
incurs additional false negatives due to statistical error.

\medskip 
The general Julia developer will be interested in: 
\begin{itemize}
    \item The main causes of a violation of strong parallelism invariance that we have identified 
    (\cref{sec:PI_causes})---some of which are specific to Julia---and how we address 
    them in Pigeons.jl.

    \item The SplittableRandoms.jl\footnote{\url{https://github.com/Julia-Tempering/SplittableRandoms.jl}}
    package that was developed by our team 
    to achieve strong parallelism invariance in Pigeons.jl (\cref{sec:splittable_randoms}).
\end{itemize}
